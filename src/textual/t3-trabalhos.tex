\chapter{Trabalhos Relacionados}
\label{cap:trabalhos}
Lorem ipsum dolor sit amet, consectetur adipiscing elit. Fusce sodales nibh quis ligula imperdiet sodales. Cras eleifend nulla eu elit pharetra, et lobortis risus semper. Nulla eu eros felis. Nam tincidunt velit ac velit pretium sagittis. Curabitur ultrices suscipit elit, commodo congue elit semper ut. Suspendisse dapibus sagittis tortor non euismod. Nullam fringilla arcu nunc, quis imperdiet ante condimentum vel. Nunc ornare leo nibh, ac pretium mauris porta vel. Nulla rhoncus, velit ac commodo blandit, tellus odio viverra erat, ac vehicula justo eros non ligula. Curabitur pretium aliquet eros, a maximus tellus suscipit nec. 
%%%%%%%%%%%%%%%%%ALTERADO a estrutura, criado subsection %%% %%%%%%%%%%%%%%%%%%%%

\section{Aplicativo...}

Os trabalhos pesquisados foram selecionados com base nas seguintes abordagens: sistemas web e \textit{mobile}, geolocalização e informações geográficas. Dentre os trabalhos selecionados dois se destacam, por estarem mais próximos dos objetivos desta monografia.

\subsection{Características..}

A Prefeitura Municipal de Juiz de Fora, disponibilizou em 2016 um aplicativo \textit{beta web} chamado Descarte.me (\autoref{fig:descarta}). Este sistema, apresenta o mapa exibindo a rota do serviço, a localização dos caminhões, a estimativa da passagem do veículo na rua e os dias e horários da coleta. De acordo com o prefeito, Bruno Siqueira, $``$As pessoas poderão ver quanto tempo falta para a próxima coleta, e o lixo poderá ser colocado para fora em um horário mais próximo, evitando que as ruas fiquem sujas e que o lixo seja revirado$"$  \cite{tribuna}.

%%%%%%%%%%%%%%%%%%%% Figure/Image No: 6 starts here %%%%%%%%%%%%%%%%%%%%

\begin{figure}[H]
	\begin{Center}
		\includegraphics[width=0.95in,height=1.75in]{./media/image37.png}
	\end{Center}
	\caption{Tela Inicial do Aplicativo Descarte.me}
	\label{fig:descarta}
	\source{\cite{tribuna}.}
\end{figure}

%%%%%%%%%%%%%%%%%%%% Figure/Image No: 6 Ends here %%%%%%%%%%%%%%%%%%%%



O trabalho citado acima difere-se deste projeto, nos seguintes pontos (\autoref{qua:descarta}):

%%%%%%%%%%%%%%%%%%%% Table No: 2 starts here %%%%%%%%%%%%%%%%%%%%

\begin{quadro}[H]
\caption{Quadro Comparativo do Aplicativo Descarte.me}
\label{qua:descarta}
\centering
\begin{tabular}{p{1.18in}p{2.3in}p{2.09in}}
\hline
%row no:1
\multicolumn{1}{|p{1.18in}}{\Centering \textbf{Características}} & 
\multicolumn{1}{|p{2.3in}}{\Centering \textbf{Descarta.me}} & 
\multicolumn{1}{|p{2.09in}|}{\Centering \textbf{Este Projeto}} \\
\hhline{---}
%row no:2
\multicolumn{1}{|p{1.18in}}{\textbf{Plataforma}} & 
\multicolumn{1}{|p{2.3in}}{Web \par Android  \par iOS} & 
\multicolumn{1}{|p{2.09in}|}{Android} \\
\hhline{---}
%row no:3
\multicolumn{1}{|p{1.18in}}{\textbf{Tecnologias}} & 
\multicolumn{1}{|p{2.3in}}{GPS \par HTML \par APIs Google Maps} & 
\multicolumn{1}{|p{2.09in}|}{Android Studio \par APIs Google Maps \par Java \par Firebase \par Firebase Realtime Database \par GPS} \\
\hhline{---}
%row no:4
\multicolumn{1}{|p{1.18in}}{\textbf{Funcionalidades}} & 
\multicolumn{1}{|p{2.3in}}{Localização dos caminhões \par Previsão da passagem dos veículos  \par Dias e horários da coleta \par Rota } & 
\multicolumn{1}{|p{2.09in}|}{Cadastro de perfil de usuários \par Rota informando a distância e a duração \par Solicitação do serviço da coleta} \\
\hhline{---}

\end{tabular}
\end{quadro}
%%%%%%%%%%%%%%%%%%%% Table No: 2 ends here %%%%%%%%%%%%%%%%%%%%

No \autoref{qua:descarta} podemos observar os pontos diferentes entre o projeto Descarta.me e este projeto. Neste projeto teremos como vantagem a implementação do \textit{Firebase}, do \textit{Firebase Realtime Database} como banco de dados, da rota e da solicitação do serviço de coleta via aplicativo.

\subsection{Aplicativo Cataki}

 O aplicativo Cataki foi criado pelo projeto \textit{Pimp My Carroça}, lançado em 2017, destinado para as seguintes plataformas móveis: \textit{Android }e \textit{iOS. }Seu objetivo é conectar catadores ou geradores de resíduos. O aplicativo não possui fins lucrativos e nem colaborativos. 

 No aplicativo é possível cadastrar catadores e cooperativas via \textit{site}, localizá-los através de um mapa, onde o catador é representado pelo ícone de uma carroça (\autoref{fig:cataki}). O mapa apresenta detalhes do coletor, como por exemplo: sua biografia, telefone, endereço, bairros que realiza as coletas e tipos de materiais. Para solicitar o serviço da coleta, é preciso entrar em contato com o coletor via mensagem ou ligação.




%%%%%%%%%%%%%%%%%%%% Figure/Image No: 7 starts here %%%%%%%%%%%%%%%%%%%%

\begin{figure}[H]
	\begin{Center}
		\includegraphics[width=1.75in,height=2.79in]{./media/image51.png}
	\end{Center}
	\caption{Tela Inicial do Aplicativo}
	\label{fig:cataki}
	\end{figure}

%%%%%%%%%%%%%%%%%%%% Figure/Image No: 7 Ends here %%%%%%%%%%%%%%%%%%%%

 O trabalho citado acima difere-se deste projeto, nos seguintes pontos (\autoref{quad:cataki}):


Pode-se observar que no ...
