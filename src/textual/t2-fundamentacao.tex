\chapter{FUNDAMENTAÇÃO TEÓRICA}
\label{cap:fundamentacao}

\lipsum[5] %Apagar esta linha - Gerador de texto aleatório

\section{Estereologia} 

A possibilidade de quantificar as características microestruturais de materiais vem se desenvolvendo desde os meados do século XIX \cite{Friel2000, Mannheimer2002}. As propriedades típicas e o comportamento mecânico dos materiais na sua maioria são influenciados pelas suas microestruturas. Nas ligas metálicas por exemplo, a microestrutura é caracterizada pelo número de fases presentes, suas proporções e a maneira na qual estas são distribuídas. Dai vem o interesse de engenheiros em investigar e caracterizar as microestruturas de seus objetos de estudo \cite{Callister2012}.

\lipsum[6] %Apagar esta linha - Gerador de texto aleatório

Depois de mais de dois séculos desde sua origem, a estereologia é parte integrante de muitos campos científicos, como a mineralogia, metalografia, biologia, dentre outros \cite{Miyamoto1994, Russ1986, Friel2000, Baddeley2004, Mannheimer2002}.

 A relação entre fração linear (\gls{LL}) e fração de volume (\gls{VV}) foi demonstrada pelo geólogo alemão \apudonline{Rosiwal1898}{Baddeley2004}. Este pesquisador demonstrou que a soma dos comprimentos dos segmentos de linha dentro da fase de interesse dividida pelo comprimento total forneceria uma estimativa válida da fração de volume com menos esforço do que a análise por área \cite{Pabst2015, Baddeley2004,Friel2000, Russ1986}.
 
\lipsum[6] %Apagar esta linha - Gerador de texto aleatório
 
O trabalho de \citeonline{Pessanha2016}...

Exemplos de citações a sites \cite{Benjamin2014, Plugins2018}...