
\chapter{RESULTADOS E DISCUSSÕES}
%%%%%%%%%%%%%%%%%ALTERADO%%% %%%%%%%%%%%%%%%%%%%%
\label{cap:resultados}
 Neste capítulo são apresentados os resultados obtidos após o desenvolvimento do protótipo e a análise dessas informações. O desenvolvimento deste projeto cumpriu as etapas planejadas e teve como resultado um protótipo com dois perfis: O perfil catador, o qual é responsável pelo recolhimento do material e o perfil separador, que solicita o serviço da coleta de resíduos sólidos.

No final do testes foi coletado o questionário de avaliação dos participantes com os itens avaliados, sendo estas informações representadas na \autoref{fig:resultados}.

\begin{figure}[H]
	\begin{Center}
		\includegraphics[width=5.0in,height=3.06in]{./media/resultados.png}
	\end{Center}
\caption{Gráfico dos resultados obtidos do questionário}
\label{fig:resultados}
\end{figure}

Na \autoref{fig:resultados} citada acima foi possível observar que foram criados quatro perfis de usuário: dois perfis separadores e dois perfis catadores. A média dos resultados das perguntas foi satisfatória.

O \autoref{cod:cadbanc0} é responsável por salvar o usuário catador no \textit{firebase realtime database}. A linha 3 do \autoref{cod:cadbanc0}, consiste na criação do nó (\textit{child}) denominado “Catadores”, onde cada catador possui um identificador (\textit{id}), onde seus dados são salvos (\textit{setValue}).

\begin{codigo}[H]
\begin{lstlisting}[language=Java]
public void salvar(){
    DatabaseReference databaseReference = ConfiguracaoFirebase.getFirebase();
    databaseReference.child("Catadores").child(getId()).setValue(this);
}
\end{lstlisting}
\caption{Salvar usuário catador no banco de dados}
\label{cod:cadbanc0}
\end{codigo}


Na \autoref{fig:login} a seguir, é apresentada a tela inicial da aplicação. 
%%%%%%%%%%%%%%%%%%%% Figure/Image No: 54 starts here %%%%%%%%%%%%%%%%%%%%

\begin{figure}[H]
	\begin{Center}
		\includegraphics[width=1.85in,height=2.94in]{./media/image34.png}
	\end{Center}
\caption{Tela de Login}
\label{fig:login}
\end{figure}
%%%%%%%%%%%%%%%%%%%% Figure/Image No: 54 Ends here %%%%%%%%%%%%%%%%%%%%
Na tela inicial (\autoref{fig:login}) da aplicação citada acima, o usuário já cadastrado irá informar seu \textit{e-mail} e senha e realizar o \textit{login} no sistema. A aplicação permite a visualização da senha digitada e o armazenamento dos dados de autenticação.
