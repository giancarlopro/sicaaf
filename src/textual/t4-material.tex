\chapter{Metodologia}
\label{cap:metodologia}
Este capítulo apresenta as técnicas, tecnologias e ferramentas utilizadas no desenvolvimento da aplicação.

Neste trabalho foi proposto um processo para obtenção de valores mais precisos de \gls{K-IC}, e este segue o seguinte roteiro, conforme o modelo apresentado na \autoref{fig:metodologia_tese}.


\tikzstyle{decision} = [diamond, draw, fill=orange!30, 
    text width=4.5em, text badly centered, node distance=3cm, inner sep=0pt]
\tikzstyle{block} = [rectangle, draw, fill=blue!20, 
    text width=6em, text centered, rounded corners, minimum height=4em]
\tikzstyle{line} = [draw, -latex']
\tikzstyle{cloud} = [draw, ellipse,fill=red!20, node distance=3cm,
    minimum height=2em]

\tikzstyle{startstop} = [rectangle, rounded corners, minimum width=3cm, minimum height=1cm,text centered, draw=black, fill=red!30]
\tikzstyle{process} = [rectangle, minimum width=3cm, minimum height=1cm, text centered, draw=black, fill=blue!20]

\begin{figure}[H]
\centering
\resizebox{!}{15cm} {
\begin{tikzpicture}[node distance = 1.5cm, auto]
    % Place nodes
    \node [startstop] (init) {Determinação de $K_{IC}$};
    %node [block] (init) {determinação de $K_{IC}$};
    \node [process, below of=init] (confeccaoL) {Confecção CPL p/ compressão};
    \node [process, below of=confeccaoL] (compressao) {Ensaio de compressão};
    \node [process, below of=compressao] (propriedades) {Determinação de E e $\nu$};
    \node [process, below of=propriedades] (confeccaoF) {Confecção CP flexão};
    \node [process, below of=confeccaoF] (flexao) {Ensaio de flexão CP liso};
    \node [process, below of=flexao] (kic_preliminar) {Ensaios $K_{IC}$ preliminares};
    \node [process, below of=kic_preliminar] (tdc) {Estimativa $\rho_{min}$ pela TDC};
    \node [decision, below of=tdc, node distance=2.3cm] (decisao) {Comport. Trinca?};
    \node [process, left of=decisao, xshift=-1cm, yshift=-2cm] (gomez) {Aplicar o Critério de Gómez};
    \node [process, right of=decisao, xshift=1cm, yshift=-2cm] (curvaR) {Corrigir com Curva-R};
    \node [process, below of=gomez] (kuc) {$K^{U}_{C}$};
    \node [process, below of=curvaR] (kic) {$K_{IC}$};
    \node [process, below of=kuc] (kicg) {$K_{IC}^{*}$};
    \node [process, below of=kic] (kin_kmax) {$K_{IN},K_{MAX}$};
    \node [startstop, below of=kuc, xshift=2.5cm, yshift=-1.5cm] (stop) {Fim};

    % Draw edges
    \path [line] (init) -- (confeccaoL);
    \path [line] (confeccaoL) -- (compressao);
    \path [line] (compressao) -- (propriedades);
    \path [line] (propriedades) -- (confeccaoF);
    \path [line] (confeccaoF) -- (flexao);
    \path [line] (flexao) -- (kic_preliminar);
    \path [line] (kic_preliminar) -- (tdc);
    \path [line] (tdc) -- (decisao);
    \path [line] (decisao) -- node {não} (gomez); % (gomez)+(-0.3,0.9)
    \path [line] (decisao) -- node {sim} (curvaR);
    \path [line] (gomez)  -- (kuc);
    \path [line] (curvaR) -- (kic);
    \path [line] (kuc)  -- (kicg);
    \path [line] (kic) -- (kin_kmax);
    \path [line] (kicg)  -- (stop);
    \path [line] (kin_kmax) -- (stop);
\end{tikzpicture}
}
\caption{Metodologia para determinação de $K_{Ic}$ desta Tese.}
\label{fig:metodologia_tese}
\end{figure}



\section{Etapas da Pesquisa}


A \glsdesc{a-0} e o \glsdesc{rho}, foram digitados na planilha, visando o cálculo da \glsdesc{K-I}, utilizando a Equações \ref{eq:tenacidade-Tada} e \ref{eq:F-a-h-Tada},

\begin{equation}
\label{eq:tenacidade-Tada}
K_{I} = \sigma_{max} \sqrt{\pi a} \cdot F(a/h)
\; ,
\end{equation}
\begin{equation}
\label{eq:F-a-h-Tada}
\selectlanguage{spanish}
F(a/h) = 1.122 - 1.40 (a/h) + 7.33 (a/h)^{2} - 13.08(a/h)^{3} + 14.0 (a/h)^{4}
\; ,
\end{equation}
\begin{conditions}
$\glssymbol{K-I}$			& \text{\glsdesc{K-I}} \\
$\glssymbol{sigma-max}$		& \text{\glsdesc{sigma-max}} \\
$\glssymbol{a}$				& \text{\glsdesc{a}} \\
$\glssymbol{h}$				& \text{\glsdesc{h}} \\
$\glssymbol{F-a-h}$			& \text{\glsdesc{F-a-h}} \\
\end{conditions}




\section{AAA}

A citação online é utilizada como parte da frase, fazendo sentido na mesma, conforme neste exemplo originalmente citado por \citeonline{kobayashi1999}. Obviamente este autor não tem nada a ver com a frase.

\section{Ferramentas}

Nesta seção é apresentada um exemplo de Tabela (\autoref{tb:weibull_02_09}) adicionada ao texto no formato \LaTeX.



\begin{table}[H]
\centering
\caption{Distribuição de Weibull com a serra de \SI{0.20}{\mm}, $a_{0} \approx \SI{9}{\mm}$}
\label{tb:weibull_02_09}
%\rotatebox{90} {
%\resizebox{!}{7cm} {
\resizebox{14cm}{!} {
\begin{tabular}{@{}rrrrrrrrr@{}}
\toprule
i & \# cp & $sigma_{nom} (MPa)$ & $PF = \dfrac{i - 0.5}{n}$ & $X = ln(\sigma_nom)$ & $Y = ln\left(ln\left(\dfrac{1}{1-Pf}\right)\right)$ & Sxx & Syy & Sxy \\ 
\midrule
1 & 209 & 4.74 & 0.02 & 1.56 & -3.90 & 0.01 & 11.13 & 0.37 \\
2 & 214 & 4.93 & 0.06 & 1.59 & -2.78 & 0.01 & 4.91 & 0.16 \\
3 & 223 & 4.96 & 0.10 & 1.60 & -2.25 & 0.00 & 2.84 & 0.11 \\
4 & 211 & 5.08 & 0.14 & 1.62 & -1.89 & 0.00 & 1.76 & 0.06 \\
5 & 202 & 5.10 & 0.18 & 1.63 & -1.62 & 0.00 & 1.11 & 0.04 \\
6 & 225 & 5.14 & 0.22 & 1.64 & -1.39 & 0.00 & 0.68 & 0.03 \\
7 & 206 & 5.15 & 0.26 & 1.64 & -1.20 & 0.00 & 0.40 & 0.02 \\
8 & 224 & 5.19 & 0.30 & 1.65 & -1.03 & 0.00 & 0.22 & 0.01 \\
9 & 204 & 5.19 & 0.34 & 1.65 & -0.88 & 0.00 & 0.10 & 0.01 \\
10 & 201 & 5.30 & 0.38 & 1.67 & -0.74 & 0.00 & 0.03 & 0.00 \\
11 & 216 & 5.30 & 0.42 & 1.67 & -0.61 & 0.00 & 0.00 & 0.00 \\
12 & 220 & 5.31 & 0.46 & 1.67 & -0.48 & 0.00 & 0.01 & 0.00 \\
13 & 219 & 5.33 & 0.50 & 1.67 & -0.37 & 0.00 & 0.04 & 0.00 \\
14 & 215 & 5.33 & 0.54 & 1.67 & -0.25 & 0.00 & 0.10 & 0.00 \\
15 & 208 & 5.34 & 0.58 & 1.67 & -0.14 & 0.00 & 0.18 & 0.00 \\
16 & 203 & 5.35 & 0.62 & 1.68 & -0.03 & 0.00 & 0.28 & 0.00 \\
17 & 222 & 5.43 & 0.66 & 1.69 & 0.08 & 0.00 & 0.41 & 0.02 \\
18 & 207 & 5.48 & 0.70 & 1.70 & 0.19 & 0.00 & 0.56 & 0.02 \\
19 & 213 & 5.51 & 0.74 & 1.71 & 0.30 & 0.00 & 0.75 & 0.03 \\
20 & 205 & 5.52 & 0.78 & 1.71 & 0.41 & 0.00 & 0.96 & 0.04 \\
21 & 221 & 5.54 & 0.82 & 1.71 & 0.54 & 0.00 & 1.22 & 0.05 \\
22 & 210 & 5.55 & 0.86 & 1.71 & 0.68 & 0.00 & 1.54 & 0.06 \\
23 & 217 & 5.56 & 0.90 & 1.72 & 0.83 & 0.00 & 1.96 & 0.07 \\
24 & 218 & 5.61 & 0.94 & 1.73 & 1.03 & 0.00 & 2.56 & 0.09 \\
25 & 212 & 5.75 & 0.98 & 1.75 & 1.36 & 0.01 & 3.72 & 0.16 \\
N = & 25 &  & Médias & 1.67 & -0.57 &  &  &  \\
 &  &  &  &  & Somas & 0.05 & 37.48 & 1.35 \\ 
\bottomrule
\end{tabular}
}
%}
\end{table}

Outro exemplo importante é o de Tabelas em LandScape, conforme a \autoref{tb:dados_cp_flexao_02_09}

Estudos envolvendo física, engenharia e matemática em geral precisam utilizar letras gregas. Tais letras podem ser utilizados dentro de equações matemáticas no \LaTeX. Quando for necessário utilizá-las em frases comuns deve-se utilizar o recurso de expressão em linha, adicionando um cifrão antes e outro depois da expressão criada, como em $\alpha , \beta , \delta, \gamma $ ...