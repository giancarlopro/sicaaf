\chapter{Metodologia}
\label{cap:metodologia}
\section{Etapas do Trabalho}
Este trabalho terá sua metodologia baseada no desenvolvimento de uma catraca automatizada com sistema próprio capaz de ler cartões de identificação com código de barras e então autorizar ou não a passagem.

Será seguido os seguintes passos a fim de validar essa catraca:
\begin{itemize}
    \item Capacidade de ler códigos de barras:
    \subitem Onde será avaliada a eficácia da catraca automatizada de reconhecer códigos de barra, utilizando leitores dos mais variados tipos.
    \item Capacidade de, através de busca automatizada, permitir ou negar o acesso ao usuário:
    \subitem Neste passo, será avaliada a capacidade do equipamento de, após receber um número do código de barras, percorrer pelo banco de dados e encontrar um indivíduo que possua aquele dado de maneira satisfatoriamente rápida. 
    \item Apresentar economia significativa em seu desenvolvimento se comparada a catracas similares disponíveis:
    \subitem Um ponto crucial desta monografia é a economia de recursos, portanto é necessário avaliar quanto investimento foi necessário para o projeto acontecer e o qual a economia relativa aos equipamentos das empresas pesquisadas. 
    \item Possibilitar customização para adaptação a diversos cenários:
    \subitem Finalmente, será avaliada a capacidade do sistema embutido de se adaptar a diversas situações. 
\end{itemize}