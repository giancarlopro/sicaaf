\chapter{Conclusão}
\label{cap:conclusão}
%%%%%%%%%%%ALTERADO%%%%%%%
Orci varius natoque penatibus et magnis dis parturient montes, nascetur ridiculus mus. Aenean nec pulvinar libero, et pellentesque elit. Sed a congue sapien. Etiam pretium ligula sit amet blandit porttitor. Cras accumsan pretium eros ut imperdiet. Donec vel lacus ante. Quisque consectetur ligula sit amet leo accumsan lobortis gravida sit amet lacus. Integer a leo nisl. Mauris pretium fringilla lobortis. Ut condimentum, lorem vitae viverra tincidunt, est mauris ornare augue, ac semper mi lorem et enim. Vestibulum venenatis id sapien vel volutpat. Pellentesque ut porta nisi. Pellentesque mattis metus a quam congue, sit amet viverra ex auctor.

Além disso, com a informação do tipo de material depositado pelo separador, o catador otimiza seu tempo, podendo ter um maior número de materiais recolhidos, e com a facilidade de escolher onde descartar o material separado, mais pessoas podem realizar a separação de materiais em suas rotinas. 

Class aptent taciti sociosqu ad litora torquent per conubia nostra, per inceptos himenaeos. Morbi pretium posuere augue nec varius. Sed ac sem in magna dapibus iaculis. Nam elementum neque eget tellus viverra scelerisque. Aenean ut posuere elit. Morbi eget erat non ante congue tempor eget id erat. Nulla quis luctus nulla. Mauris auctor vel magna sed ultrices. Integer in massa molestie, ornare urna id, viverra sem. Vivamus interdum, ligula eget convallis tempor, dui tellus tristique nibh, in interdum arcu sem a tellus. 

\section{Trabalhos Futuros}

Maecenas id nisi vel neque tincidunt faucibus quis at nibh. Aenean consequat eleifend ultrices. Cras dui nibh, luctus a nibh ultrices, condimentum gravida diam. Cras dapibus sit amet mauris at pharetra. Vestibulum quis malesuada mi. Nunc pretium, neque at placerat maximus, urna nunc hendrerit urna, eget efficitur sem ipsum vel mauris. Donec eget efficitur eros, at cursus dolor.

\begin{itemize}
	\item Acrescentar todos os pontos de descarte no mapa;
	\item Otimizar a rota através de algoritmos;
	\item Aplicar filtros ao mapa, para que o catador possa visualizar somente os pontos pelo tipo de material selecionado;
	\item Enviar notificações quando forem inseridos novos pontos de descarte e quando for sinalizado o recolhimento de um material;
\setlength{\parskip}{9.96pt}
	\item Adicionar outros métodos de \textit{login}, como por exemplo: o \textit{Facebook, }o \textit{Gmail }e o número de telefone;
	\item Inserir uma rota dinâmica no programa, em função da localização instantânea do usuário.
\end{itemize}

 \glsaddall