\chapter{INTRODUÇÃO}
\label{cap:introducao}

De acordo com Chiavenato \cite{chiavenato2003introducao} há na sociedade necessidades humanas básicas,
dentre elas a necessidade de se manter seguro com relação a ameaças externas
que o coloquem em perigo físico. A partir disso é possível perceber que a sensação
de segurança é de suma relevância para o convívio em sociedade, e como há um crescente
aumento dos aglomerados se fez necessário implementar medidas que possam assegurar a
integridade de todos, e devido ao alto custo atual dos modelos disponíveis a necessidade
de encontrar uma maneira de sanar o problema sem que se gaste muitos recursos se faz
cada vez mais evidente. É calculado então que o sucesso será atingido no momento em que
o nível de preservação física estiver atestado e o capital poupado.


Sendo assim, os institutos de ensino como um todo possuem um alto fluxo de pessoas
diariamente, sejam alunos, funcionários e/ou visitantes. Dito isso, se faz necessário
estabelecer uma forma de controlar quem circula pelas dependências dos prédios, quando
entram e/ou saem, no entanto, atualmente, os meios convencionais de solução disponíveis
no mercado são extremamente custosos o que impede a implementação na maioria dos lugares.


Desta forma, enquanto existirem espaços públicos ou privados, os quais possuam ou não
informações sensíveis, onde transitem um número considerável de indivíduos, haverá o
desejo e a necessidade de controlar a movimentação de todos.

\section{Objetivos}

Busca-se então explorar medidas de controle de custo reduzido as quais tornarão
possível a implementação de um sistema confiável que auxiliará na segurança e que,
quando aplicadas corretamente, serão benéficas para todas as pessoas que circulam
por estas instituições. Capaz de identificar e autorizar a entrada e saída daqueles
inseridos no espaço em questão, desenvolver uma aplicação que execute com o mínimo
de erros possível aquilo que for estabelecido na fase de projeto e implementar de
forma íntegra o software desenvolvido, tendo como perspectiva de resultado positivo
o aumento na segurança e a economia relativa ao uso de tecnologia similar já existente
no cenário tecnológico atual.

\section{Justificativa}

O projeto se baseia na dificuldade atual de se implantar sistemas similares,
uma vez que as tecnologias que se encontram no mercado são muito custosas e,
na maioria dos casos, inacessíveis ao grande público. Segundo matéria veiculada
no portal online do Instituto Federal Fluminense, “Fabricação de catracas por
alunos irá trazer economia de até R\$ 870 mil ao IFF Campos Centro”, publicada
em 24/05/2019, estudantes avaliaram orçamentos de diversas empresas do ramo permitindo
estabelecer um escopo que varia entre trezentos e oitocentos e setenta mil reais,
enquanto o orçamento para este projeto é de sessenta mil reais, para a instalação
das dezenove catracas e de um sistema que as gerencie.


Sendo assim, no momento em que o sistema atingir seu estado de produção será possível
aplicá-lo a qualquer ambiente que necessite de um controle genérico de acesso, mas
principalmente em escolas e universidades que queiram realizar o
controle do fluxo de estudantes.


\section{Estrutura do Trabalho}
% TODO: This section
Este trabalho está dividido em sete capítulos.
O \autoref{cap:introducao} expõe o contexto do estudo,
% as justificativas desta pesquisa e os objetivos a serem atingidos.
% O \autoref{cap:fundamentacao} apresenta conceitos de caracterização de
% materiais, relata algumas normas de organizações internacionais para este
% fim e faz uma explanação sobre a área de processamento e análise de imagens.
% O \autoref{cap:trabalhos}... . Finalmente, o \autoref{cap:conclusão} apresenta
% a conclusão e os potenciais trabalhos futuros a serem desenvolvidos.
