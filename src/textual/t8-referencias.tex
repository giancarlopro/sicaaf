
\cite{WANDRESEN:2016}
WANDRESEN, Rafael Romualdo \textit{et al.} SOFTWARE EDUCACIONAL PARA O ENSINO DE INVENTÁRIO FLORESTAL. \textbf{BIOFIX Scientific Journal}, v.1, n. 1, 2016.





ANATEL. \textbf{Brasil registra 231,8 milhões de linhas móveis em novembro}. 2019. Disponível em: http://www.anatel.gov.br/institucional/noticias-destaque/2175-brasil-registra-231-8-milhoes-de-linhas-moveis-em-novembro. Acesso em: 29 de jan. de 2019.





TRIBUNA. \textbf{Prefeitura lança app para otimizar coleta de lixo}. 2016. Disponível em: https://tribunademinas.com.br/noticias/cidade/20-07-2016/prefeitura-lanca-app-para-otimizar-coleta-de-lixo.html. Acesso em: 12 de nov. de 2018





FIREBASE. \textbf{Produtos}. 2019. Disponível em: https://firebase.google.com/products/?hl=pt-br. Acesso em: 28 de jan. de 2019.





DB-ENGINES. \textbf{DB-Engines Ranking of Document Stores}. 2019. Disponível em: https://db-engines.com/en/ranking/document+store. Acesso em: 27 de jan. de 2019. 





SILVA, Bruno de Castro Honorato. \textbf{Otimização de Rotas Utilizando Abordagens heurísticas em um Ambiente Georreferenciado}.\ Dissertação (Mestrado em Ciência da Computação), Universidade Estadual do Ceará - UECE, Fortaleza,  2013.





OLIVEIRA, Tiago Marques. \textbf{Aplicação Android de tracking e recomendação automática de rotas}. Dissertação (Mestrado em Engenharia Informática), Instituto Superior de Engenharia do Porto - ISEP, Porto, 2018.





LUMMERTZ, Ramon Santos; SGANZERLA, Antoni. Direto ao Ponto-App colaborativo do transporte coletivo usando o Firebase. \textbf{Conversas Interdisciplinares}, v. 14, n. 1, 2018.





SAREEN, Pankaj. Cloud computing: types, architecture, applications, concerns, virtualization and role of it governance in cloud. \textbf{International Journal of Advanced Research in Computer Science and Software Engineering}, v. 3, n. 3, 2013. \\




FLING, Brian. \textbf{Mobile design and development}: Practical concepts and techniques for creating mobile sites and Web apps. O’ Reilly Media, Inc., 2009. 





 BURTON, Michael; FELKER, Donn. \textbf{Desenvolvimento de Aplicativos Android para Leigos}. 2. ed. Rio de Janeiro: Alta Books Editora, 2014.





PANIZ, David. \textbf{NoSQL}: Como armazenar os dados de uma aplicação moderna. Editora Casa do Código, 2016.





ELMASRI, Ramez; NAVATHE, Shamkant B. \textbf{Sistemas de Banco de Dados}. 6. ed. São Paulo: Pearson, 2011.




PETROSKI, Henry. \textbf{Success through failure:} The paradox of design. Princeton University Press, 2018.





VIEIRA, Marcos Rodrigues\tab  \textit{et al}. Banco de dados NoSQL:\textbf{ }Conceitos, Ferramentas, Linguagens e Estudos de Casos no Contexto de Big Data. \textit{In}: \textbf{Simpósio Brasileiro de Bancos de Dados,} 2012.





MENZORI, Mauro. \textbf{Georreferenciamento-Conceitos}. Editora Baraúna, 2017.





SANTOS, Sandro R. \textbf{Aplicativos Móveis um Negócio Rentável}: Tudo sobre como Ganhar Muito Dinheiro Criando Apps. SSTrader Editor, 2018.\\



GARTNER. \textbf{Gartner Says Huawei Secured No. 2 Worldwide Smartphone Vendor Spot, Surpassing Apple in Second Quarter 2018}. 2018. Disponível em: https://www.gartner.com/en/newsroom/press-releases/2018-08-28-gartner-says-huawei-secured-no-2-worldwide-smartphone-vendor-spot-surpassing-apple-in-second-quarter. Acesso em: 13 de ago. de 2018.





HELLMAN, Rafael Alexandre Freiberger. \textbf{Aplicativo android e website interativos para busca de menores preços de produtos com código de barras}. Monografia (Graduação em Engenharia Elétrica com ênfase em Eletrônica) - Escola de Engenharia de São Carlos - EESC, Universidade de São Paulo - USP, São Carlos, 2016. \\
 \\





NETO, Manoel Silva. \textbf{Georreferenciamento de imóveis rurais com drones}:\ entenda as mudanças. 2018. Disponível em: http://blog.droneng.com.br/georreferenciamento-de-imoveis-rurais-com-drones/. Acesso em: 01 de out. de 2018.  \\




MMA. \textbf{Crianças aprendem sobre coleta seletiva}. 2018. Disponível em: http://www.mma.gov.br/informma/item/14853-noticia-acom-2018-06-3083.html. Acesso em: 26 de jul. de 2018.


MILANI, André. \textbf{Programando para iPhone e iPad}: Aprenda a construir aplicativos para o iOS. 2. ed. São Paulo: Novatec Editora, 2014.





LOESCHE, Dyfed. \textbf{The Biggest App Stores}. 2018. Disponível em: https://www.statista.com/chart/12455/number-of-apps-available-in-leading-app-stores/. Acesso em: 14 de ago. de 2018.





IDC. \textbf{Smartphone Market Share}. 2019. Disponível em: https://www.idc.com/promo/smartphone-market-share/os. Acesso em: 21 de fev. de 2019.\\
 \\

LECHETA, Ricardo R. \textbf{Desenvolvendo para iPhone e iPad}: Aprenda a desenvolver aplicativos utilizando iOS SDK. 5. ed. São Paulo: Novatec Editora, 2017.




DE MENEZES, João Paulo. \textbf{Protótipo de Sistema para Monitoramento em Tempo Real de Produtos e Veículos}. Monografia (Bacharelado em Ciência da Computação), Instituto Federal de Educação, Ciência e Tecnologia de Minas Gerais - IFMG, Formiga, 2017. 





MEYER, Maximiliano. \textbf{A história do iOS [atualizado iOS 12.1]}.\ 2018.  Disponível em: https://www.oficinadanet.com.br/post/17950-a-historia-do-ios. Acesso em: 01 de nov. de 2018.



Nunes, João Pedro Araújo. \textbf{ONU estima que lixo produzido no mundo será 70$\%$  maior em 2030}. 2015. Disponível em: https://agenotic.wordpress.com/2015/01/28/lixo-mundo. Acesso em: 04 de set. de 2017.\\

PIZARRO, Ludmila. \textbf{Brasil perde R$\$$ 120 bilhões por ano ao não reciclar lixo}. 2017. Disponível em: http://www.otempo.com.br/capa/economia/brasil-perde-r-120-bilh$\%$ C3$\%$ B5es-por-ano-ao-n$\%$ C3$\%$ A3o-reciclar-lixo-1.1423628. Acesso em: 04 de set. de 2017.\\
 \\
SANTOS, Rafael. \textbf{Introdução a programação Orientada a Objetos usando Java.}\ 2. ed.  Rio de Janeiro: Elsevier, 2013.\\
 \\
REVISTA ECOTURISMO.\textbf{ Lixo Eletrônico, uma Oportunidade de Transformação e Crescimento para o Brasil}. 2015. Disponível em: http://revistaecoturismo.com.br/turismo-sustentabilidade/lixo-eletronico-uma-oportunidade-de-transformacao-e-crescimento-para-o-brasil. Acesso em: 06 de set. de 2017.\\
 \\
SOUSA, Andreia Patrícia Ferreira. \textbf{Um Algoritmo Genético Para O Planeamento De Rotas Com Considerações Ambientais}. Dissertação (Mestrado em Modelação, Análise de Dados e Sistemas de Apoio à Decisão), Faculdade de Economia do Porto - FEP, Universidade do Porto, Portugal, 2015.\\
 \\
 \\
GOOGLE CLOUD. \textbf{Trajetos. }2019. Disponível em: https://cloud.google.com/maps-platform/routes/. Acesso em: 22 de fev. de 2019.





GOOGLE CLOUD. \textbf{Mapas. }2018. Disponível em: https://cloud.google.com/maps-platform/maps/. Acesso em: 22 de set. de 2018.\\

