
\chapter{Questionário de Avaliação do Protótipo}

\begin{figure}[H]
	\begin{Center}
		\includegraphics[width=6.00in]{media/questionariopaint.png}
		\label{fig:questionario}
	\end{Center}
\end{figure}

\chapter{Códigos fonte comentados}

\section{Salvar usuário catador no banco de dados}

O \autoref{cod:cadbanc} é responsável por salvar o usuário catador no \textit{firebase realtime database}. A linha 3 do \autoref{cod:cadbanc}, consiste na criação do nó (\textit{child}) denominado “Catadores”, onde cada catador possui um identificador (\textit{id}), onde seus dados são salvos (\textit{setValue}).

\begin{codigo}[H]
\begin{lstlisting}[language=Java]
public void salvar(){
    DatabaseReference databaseReference = ConfiguracaoFirebase.getFirebase();
    databaseReference.child("Catadores").child(getId()).setValue(this);
}
\end{lstlisting}
\caption{Salvar usuário catador no banco de dados}
\label{cod:cadbanc}
\end{codigo}


\section{Salvar usuário separador no banco de dados}

\begin{codigo}[H]
	\begin{lstlisting}[language=Java]
public void salvar(){
    DatabaseReference databaseReference = ConfiguracaoFirebase.getFirebase();
    databaseReference.child("Separadores").child(getId()).setValue(this);
}
   	\end{lstlisting}
   	\caption{Salvar usuário separador no banco de dados}
   	\label{cod:cadsep}
\end{codigo}

O \autoref{cod:cadsep} é responsável por salvar o usuário separador no \textit{firebase realtime database}. A linha 3 do \autoref{cod:cadbanc}, consiste na criação do nó (\textit{child}) denominado “Separadores”, onde cada separador possui um identificador (\textit{id}), onde seus dados são salvos (\textit{setValue}).


\section{Cadastrar usuário catador}

\begin{codigo}[H]
	\begin{lstlisting}[language=Java]
private void cadastrarCatador(){
    firebaseAuth = ConfiguracaoFirebase.getFirebaseAuth();
    firebaseAuth.createUserWithEmailAndPassword(
            catador.getEmail(),
            catador.getSenha()
    ).addOnCompleteListener(CadastroCatador.this, new OnCompleteListener<AuthResult>() {
        @Override
        public void onComplete(@NonNull Task<AuthResult> task) {
            if(task.isSuccessful()){
                Toast.makeText(CadastroCatador.this,"Sucesso ao cadastrar catador",Toast.LENGTH_LONG).show();
                FirebaseUser firebaseUser = task.getResult().getUser();
                catador.setId(firebaseUser.getUid());
                catador.salvar();
                startActivity(new Intent(CadastroCatador.this, Login.class));
       	\end{lstlisting}
       	\caption{Cadastrar usuário catador na aplicação}
       	\label{cod:cat}
\end{codigo}

O \autoref{cod:cat} consiste no cadastro do usuário catador. A função \textit{createUserWithEmailAndPassword}, que se encontra na linha 3, é responsável pela criação da autenticação dos usuário através do \textit{e-mail} e a senha. Em caso de sucesso, o sistema irá exibir a seguinte mensagem “Sucesso ao cadastrar catador” e o usuário será redirecionado para a tela de \textit{login}.

\section{Cadastrar usuário separador}

\begin{codigo}[H]
	\begin{lstlisting}[language=Java]
private void cadastrarSeparador(){
    firebaseAuth = ConfiguracaoFirebase.getFirebaseAuth();
    firebaseAuth.createUserWithEmailAndPassword(
            separador.getEmail(),
            separador.getSenha()
    ).addOnCompleteListener(CadastroSeparador.this, new OnCompleteListener<AuthResult>() {
        @Override
        public void onComplete(@NonNull Task<AuthResult> task) {
            if(task.isSuccessful()){
                Toast.makeText(CadastroSeparador.this,"Sucesso ao cadastrar separador",Toast.LENGTH_LONG).show();
                FirebaseUser firebaseUser = task.getResult().getUser();
                separador.setId(firebaseUser.getUid());
                separador.salvar();
                startActivity(new Intent(CadastroSeparador.this, Login.class));
       	\end{lstlisting}
       	\caption{Cadastrar usuário separador}
       	\label{cod:sep}
\end{codigo}

O \autoref{cod:sep} consiste no cadastro do usuário separador. A função \textit{createUserWithEmailAndPassword}, que se encontra na linha 3, é responsável pela criação da autenticação do usuário através do \textit{e-mail} e a senha.