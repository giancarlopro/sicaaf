\begin{resumo}[Abstract]
 \begin{otherlanguage*}{english}

% O comando lipsum abaixo é um gerador automático de texto, e deve ser substituído pelo seu abstract, isto é, o seu resumo traduzido para a lingua inglesa.

Human security is a primary need in the social body. Therefore, the search for accessible solutions capable of controlling the flow of people in environments that contain sensitive information such as: educational institutions, executive buildings, residential condominiums, corporate complexes and others is growing. A widely used solution is the use of turnstiles, whether automated or not, using different types of technology, such as radio frequency, which associated with the concept of “Internet of Things”, that impels data to be collected and / or processed, allow that primordiality to be ensured. That said, several institutions are analyzing the use of computerized systems that can contribute to remedy this problem with the least possible use of resources, whether human or monetary. However, the technologies offered in the national scenario, as will be exposed in the course of this monograph, are extremely costly and barely applicable when there is not a large availability of capital or trained personnel. Thus, this work aims to present the prototype of a functional system capable of guarantee this demand, making use of modern technologies which can be practiced in the most varied scenarios exemplified above. Starting from the development to the implementation of the software in a federal educational institution that will serve as a scenario for tests, where comparisons presented will demonstrate significant savings in resources and how this method can be used effectively.

\textbf{Keywords: } Turntiles, Internet of Things, System, Access Control, Software Development.
 \end{otherlanguage*}
\end{resumo}
