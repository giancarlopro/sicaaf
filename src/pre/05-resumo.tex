\begin{resumo}

% O comando lipsum abaixo é um gerador automático de texto.
% Substitua-o pelo texto do seu resumo.
% Lembre-se: Um resumo deve ser um parágrafo único que apresente os seguintes tópicos:

% Contexto;
% Problema;
% Objetivo;
% Justificativa;
% Metodologia;
% Resultado;
% Conclusão.

A segurança humana é necessidade primordial no corpo social. Sendo assim, a busca por soluções acessíveis capazes de controlar o fluxo de pessoas em ambientes que contêm informações sensíveis como: instituições de ensino, prédios executivos, condomínios residenciais, complexos corporativos e outros é crescente. Uma solução amplamente utilizada é o uso de catracas, sejam elas automatizadas ou não, podendo utilizar diferentes tipos de tecnologia, como por exemplo a rádio frequência, que associadas ao conceito de “Internet das Coisas”, o qual impele que dados sejam coletados e/ou processados, permitem que seja assegurada essa primordialidade. Dito isso, diversas instituições analisam o uso de sistemas informatizados que possam contribuir para remediar este problema com o menor uso de recursos, sejam eles humanos ou monetários, possível. No entanto, as tecnologias ofertadas no mercado, como será exposto no decorrer desta monografia, são extremamente custosas e pouco aplicáveis quando não há grande disponibilidade de capital ou pessoal treinado. Desse modo, este trabalho visa apresentar o protótipo de um sistema funcional capaz de atender a esta demanda, fazendo uso de tecnologias modernas as quais podem ser praticadas nos mais variados cenários exemplificados acima. Partindo desde o desenvolvimento à implementação do software em uma instituição federal de ensino que servirá de cenário para testes, onde comparações apresentadas demonstrarão economia significativa dos recursos e como este método pode ser utilizado com eficácia. 

\textbf{Palavras-chaves: } Catracas, Internet das Coisas, Sistema, Controle de Acesso, Desenvolvimento de Sistemas.

\end{resumo}
